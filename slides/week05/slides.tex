\documentclass[10pt,notes=hide]{beamer}
\input{../week00/slides_settings.tex}
\setbeamertemplate{footline}{\begin{center}\textcolor{gray}{Dingel -- International Macroeconomics and Trade -- Week 5 -- \insertframenumber}\end{center}}
\begin{document}
% -----------------------------------------
%TITLE FRAME
\begin{frame}[plain]
\begin{center}
\large
\textcolor{maroon}{BUSN 33946 \& ECON 35101\\
International Macroeconomics and Trade\\ 
Jonathan Dingel\\
Autumn \the\year, Week 5}
\vfill 
\includegraphics[width=0.5\textwidth]{../images/chicago_booth_logo}
\end{center}
\end{frame}
% -----------------------------------------
\begin{frame}{Today: Multiple factors of production}
With multiple factors of production, we can talk about distributional consequences of trade
\begin{itemize}
	\item Ricardo-Viner specific-factors model
	\item $2 \times 2$ Heckscher-Ohlin model
	\item High-dimensional factor-proportions theory
	\item Trade and regional outcomes (empirics with R-V lens)
	\item Heckscher-Ohlin-Vanek empirics
\end{itemize}
\end{frame}
% -----------------------------------------
\begin{frame}{Factor proportions theory}
\begin{itemize}
	\item The law of comparative advantage establishes the relationship between relative autarky prices and trade flows
	\item Factor proportions theory is an account of factor endowments as the source of relative autarky prices
	\begin{enumerate}
		\item Countries differ in terms of factor abundance (relative factor supply)
		\item Goods differ in terms of factor intensity (relative factor demand)
	\end{enumerate}
	\item The interplay between these differences governs relative autarky prices and hence trade
\end{itemize}
\end{frame}
% -----------------------------------------
\begin{frame}{Factor proportions theory}
\begin{itemize}
	\item To focus on factor endowments, shut down other channels:
	\begin{itemize}
		\item Identical production functions (no Ricardian forces)
		\item Identical homothetic preferences
	\end{itemize}
	\item Two canonical models:
	\begin{itemize}
		\item Ricardo-Viner model with 2 goods and 3 factors (2 of which are specific to a good)
		\item Heckscher-Ohlin model with 2 goods and 2 factors
	\end{itemize}
	\item \href{https://www.jstor.org/stable/pdf/2232049.pdf}{Neary (1978)}, among others, treats the specific-factors model as a short-run case, whereas all factors are mobile in longer run
\end{itemize}
\end{frame}
% -----------------------------------------
\begin{frame}{Ricardo-Viner model: Environment}
\begin{itemize}
	\item Two goods ($g=1,2$) with prices $p_1,p_2$
	\item Three factors with endowments $L$, $K_1$, $K_2$ and prices $w,r_1,r_2$
	\item Output of good $g$ is
	\begin{equation*}
		y_g = f^g \left(L_g,k_g\right)
	\end{equation*}
	where $L_g$ is (endogenous) labor working in $g$ and $f^g$ is HD1
	(payments to specific factors under CRS are profits in DRS)
	\item Profit maximization (where $f^g_{L} \equiv \pdv{f^g}{L_g}$):
	\begin{equation*}
		p_g f^g_{L} \left(L_g,K_g\right) = w 
		\qquad
		p_g f^g_{K_g}\left(L_g,K_g\right) = r_g 
	\end{equation*}
	\item Labor market clearing: $$L = L_1 + L_2$$
	\item ``Small open economy'': $p_1,p_2$ exogenous
\end{itemize}
\end{frame}
% -----------------------------------------
\begin{frame}{Ricardo-Viner model: Equilibrium}
Combine the expressions for MRPL and $L = L_1 + L_2$ to solve:
\begin{center}\includegraphics[width=.7\textwidth]{../images/Feenstra2015_fig3_2.png}\end{center}
See pages 71-75 of Feenstra textbook (first edition)
\end{frame}
% -----------------------------------------
\begin{frame}{Ricardo-Viner model: Comparative statics}
Suppose that $p_1$ increases
\\ %\begin{center}
\includegraphics[width=.55\textwidth]{../images/Feenstra2015_fig3_3.png}
%\end{center}
\begin{itemize}
	\item $w$ increases, $L_1$ increases, $L_2$ decreases
	\item Differentiate $w=p_1 f^1_{L} \left(L_1,K_1\right)$ wrt $p_1$ to find $\hat{w}<\hat{p}_1$
	\item From $p_g f^g_{K_g}\left(L_g,K_g\right) = r_g $ we know that $r_1/p_1$ rises while $r_2$ (and thus $r_2/p_1$) falls (e.g., ``Dutch disease'')
\end{itemize}
Similar exercises for changes in endowments, etc.
\end{frame}
% -----------------------------------------
\begin{frame}{$2 \times 2$ Heckscher-Ohlin model: Environment}
Production functions (HD1) using factors $L$ and $K$ are
\begin{equation*}
y_{g}=f_{g}\left( L_{g},K_{g}\right) \quad g=1,2
\end{equation*}
Unit cost functions are given by
\begin{equation*}
c_{g}\left( w,r\right) =\min_{L_{g},K_{g}}\left\{ wL_{g}+rK_{g}|f_{g}\left(
L_{g},K_{g}\right) \geq 1\right\}
\end{equation*}
We write the solution in terms of unit factor demands $a_{gf}$
\begin{equation*}
c_{g}\left( w,r\right) =wa_{gL}\left( w,r\right) +ra_{gK}\left( w,r\right)
\end{equation*}
From the envelope theorem, we know
\begin{equation*}
\frac{\textrm{d} c_{g}}{\textrm{d} w}=a_{gL} \qquad \frac{\textrm{d} c_{g}}{\textrm{d} r}=a_{gK}
\end{equation*}
$A(w,r)\equiv [a_{gf}(w,r)]$ denotes the matrix of total factor requirements
\end{frame}
% -----------------------------------------
\begin{frame}{$2 \times 2$ HO: Equilibrium in SOE}
\begin{itemize}
	\item Start with ``small open economy'' for which $p_g$ are exogenous
	\item Profit maximization:
	\begin{align*}
	p_{1} &\leq c_{1}\left( w,r\right) \quad \text{equal if produced} \\
	p_{2} &\leq c_{2}\left( w,r\right) \quad \text{equal if produced} 
	\end{align*}
	\item Labor and capital markets clear:
	\begin{align*}
	a_{1L}y_{1}+a_{2L}y_{2} &=L \\
	a_{1K}y_{1}+a_{2K}y_{2} &=K
	\end{align*}
	\item These are four nonlinear equations in four unknowns; unique solution not generally guaranteed
\end{itemize}
\end{frame}
% -----------------------------------------
\begin{frame}{Four theorems}
\begin{enumerate}
	\item Factor price equalization: Can trade in goods substitute for trade in factors?
	\item Stolper-Samuelson: Who wins and who loses from a change in goods prices?
	\item Rybczynski: How does output mix respond to change in endowments?
	\item Heckscher-Ohlin: What is the pattern of specialization and trade?
\end{enumerate}
\end{frame}
% -----------------------------------------
\begin{frame}{Factor price insensitivity}
\begin{itemize}
\item 
Good 1 is called labor-intensive if 
$\frac{a_{1L}\left(w,r\right)}{a_{1K}\left(w,r\right)}
>\frac{a_{2L}\left(w,r\right)}{a_{2K}\left(w,r\right)}$
and capital-intensive if 
$\frac{a_{1L}\left(w,r\right)}{a_{1K}\left(w,r\right)}
<\frac{a_{2L}\left(w,r\right)}{a_{2K}\left(w,r\right)}$
\item
A factor intensity reversal occurs if $\exists w,r,w',r'$ such that good 1 is labor-intensive for $(w,r)$ and capital-intensive for $(w',r')$
\end{itemize}
\begin{lemma}
If both goods are produced, and factor intensity reversals do not occur,
then factor prices $\omega \equiv (w,r)$ are uniquely determined by goods prices $p \equiv (p_1,p_2)$.
\end{lemma}
Proof:
If both goods are produced in equilibrium, then $p = A(\omega) \omega$. By Gale and Nikaido (1965), this equation admits a unique solution if $a_{fg}(\omega)>0$ for all $f,g$ and $\det[A(\omega)]\neq 0\ \forall \omega$, which no factor intensity reversals guarantees.
\end{frame}
% -----------------------------------------
\begin{frame}{Factor intensity reversals}
\includegraphics[width=.49\textwidth]{../images/Feenstra2015_fig1_5.png}
\includegraphics[width=.49\textwidth]{../images/Feenstra2015_fig1_6.png}
\end{frame}
% -----------------------------------------
\begin{frame}{Factor price equalization}
\begin{theorem}
If two countries both produce both goods under free trade with the same technology and there are no factor intensity reversals,
then factor prices in the two countries are the same.
\end{theorem}
\begin{itemize}
	\item This follow directly from the previous lemma and the no-FIR diagram:
	\begin{itemize}
		\item By free trade, goods prices are the same
		\item By identical technologies, isocost lines are the same
	\end{itemize}
	\item Hence, trade in goods is a perfect substitute for factor mobility in this model in the sense that it also equalizes factor prices
\end{itemize}
\end{frame}
% -----------------------------------------
\begin{frame}{Stolper-Samuelson Theorem}
\begin{theorem}
An increase in the relative price of one good raises the real return of the factor used intensively in producing that good and lowers the real return of the other factor.
\end{theorem}
%\begin{proof} 
Proof:\\
WLOG, let $\frac{a_{1L}(\omega)}{a_{1K}(\omega)}>\frac{a_{2L}(\omega)}{a_{2K}(\omega)}$
and $\hat{p}_{1}>\hat{p}_{2}$,
where $\hat{x} \equiv \frac{\textrm{d}x}{x}$. \\
Differentiating the zero-profit conditions yields (by envelope theorem)
\begin{equation*}
\textrm{d}p_{g}=a_{gL}\textrm{d}w+a_{gK}\textrm{d}r
\end{equation*}
Define the cost share $\theta _{gL}=\frac{wa_{gL}}{c_{g}}$
to obtain
\begin{equation*}
\hat{p}_{g}=\theta _{gL}\hat{w}+\left( 1-\theta _{gL}\right) \hat{r}
\end{equation*}
Goods price changes are weighted averages of factor price changes.
$\frac{a_{1L}}{a_{1K}}>\frac{a_{2L}}{a_{2K}} \Rightarrow \theta
_{1L}>\theta _{2L}$
so 
\begin{equation*}
\hat{r}<\hat{p}_{2}<\hat{p}_{1}<\hat{w}
\end{equation*}
%\end{proof}
\end{frame}
% -----------------------------------------
\begin{frame}{Notes on $2 \times 2$ Stolper-Samuelson Theorem}
\begin{itemize}
\item A change in product prices has a magnified effect on factor
prices
\item Jones (1965) referred to these inequalities as ``magnification
effect"
(This is the \href{https://tradediversion.net/2018/05/07/on-hat-algebra/}{original ``hat algebra''})
\item Trade liberalization that alters goods prices will thus produce winners and losers across factors
\item Like FPI and FPE, Stolper-Samuelson result follows from zero-profit condition (+ ``no joint production'')
\end{itemize}
\begin{center}
\includegraphics[width=.55\textwidth]{../images/Feenstra2015_fig1_7.png}
\end{center}
\end{frame}
% -----------------------------------------
\begin{frame}{Rybczynski Theorem}
\begin{theorem}
For given goods prices, an increase in the endowment of one factor
causes a more-than-proportionate increase in the output of the good using this factor intensively and
a decrease in the output of the other good.
\end{theorem}
Differentiating the factor market clearing conditions yields,%
\begin{equation*}
\textrm{d}L=a_{1L}\textrm{d}y_{1}+a_{2L}\textrm{d}y_{2}\text{ and\ }\textrm{d}K=a_{1K}\textrm{d}y_{1}+a_{2K}\textrm{d}y_{2}
\end{equation*}
Defining $\lambda _{iL}=\frac{a_{iL}y_{i}}{L}$ and $\lambda _{iK}=\frac{%
a_{iK}y_{i}}{K}$ this implies,%
\begin{equation*}
\hat{L}=\lambda _{1L}\hat{y}_{1}+\left( 1-\lambda _{1L}\right) \hat{y}_{2}%
\text{ and }\hat{K}=\lambda _{1K}\hat{y}_{1}+\left( 1-\lambda _{1K}\right) 
\hat{y}_{2}
\end{equation*}
If (w.l.o.g.) $\frac{a_{1L}}{a_{1K}}>\frac{a_{2L}}{a_{2K}}$, then $\lambda
_{1L}>\lambda _{1K}$ so that,%
\begin{equation*}
\hat{y}_{1}>\hat{L}>\hat{K}>\hat{y}_{2}\text{ or }\hat{y}_{1}<\hat{L}<\hat{K}%
<\hat{y}_{2}
\end{equation*}
Hence, if also (w.l.o.g.) $\hat{K}>\hat{L}$, we obtain,%
\begin{equation*}
\hat{y}_{1}<\hat{L}<\hat{K}<\hat{y}_{2}
\end{equation*}
\end{frame}
% -----------------------------------------
\begin{frame}{Rybczynski Theorem and cone of diversification}
\includegraphics[width=.37\textwidth]{../images/Feenstra2015_fig1_8.png}
\includegraphics[width=.59\textwidth]{../images/Feenstra2015_fig1_9.png}
\begin{itemize}
	\item Produce both goods iff $\left( L,K\right) $ lies between
factor requirements vectors $\left( a_{2L},a_{2K}\right) $ and $\left(
a_{1L},a_{1K}\right) $, the   ``cone of diversification"
\item At unchanged factor intensities, increase in labor endowment necessitates decrease in output of capital-intensive good
\end{itemize}
\end{frame}
% -----------------------------------------
\begin{frame}{Heckscher-Ohlin theorem}
\begin{itemize}
	\item We now consider world economy with two countries and free trade (prior results derived for small open economies)
	\item This is a $2 \times 2 \times 2$ model
	\item Identical technologies and homothetic preferences
	\item What is the pattern of trade in this global economy?
	\begin{itemize}
		\item Rather than starting from autarky, let's start from the integrated equilibrium
		\item Integrated world economy with world endowment of factors yields integrated equilibrium (good prices, factor prices, resource allocations, etc)
	\end{itemize}
\end{itemize}
\end{frame}
% -----------------------------------------
\begin{frame}{The FPE set}
\vspace{-3mm}
\begin{center}\includegraphics[width=.65\textwidth]{../images/Krugman1995_fig1_1.png}\end{center}
\vspace{-3mm}
\begin{itemize}
	\item World endowed with $K$ and $L$
	\item Integrated factor allocations $OX$ and $OY$
	\item Samuelson's angel can fragment world into two countries by endowments $E$ or $E'$.
	\item Can trade reproduce the integrated equilibrium?
	If FPE holds!
\end{itemize}
\end{frame}
% -----------------------------------------
\begin{frame}{Heckscher-Ohlin theorem}
\begin{theorem}
	In the free-trade equilibrium, each country exports the good that uses its abundant factor intensively.
\end{theorem}
\begin{itemize}
	\item If endowments are in the FPE set, this is a simple corollary of the Rybczynski theorem and homothetic preferences (no assumption on FIRs required).
	\item Outside the FPE set, need to also consider FIRs.
	\item To state the prediction in terms of autarky relative factor prices, return to general theorem of Deardorff (1980)
	\item Is the autarky relative price of the labor-intensive good lower in the labor-abundant country?
	\item See Feenstra Figure 2.1 and Jones and Neary equation (2.10)
\end{itemize}
\end{frame}
% -----------------------------------------
\begin{frame}{Higher dimensions}
What if there are $C$ countries, $G$ goods, and $F$ factors?
\begin{itemize}
	\item If $F=G$ (``even case''), situation is qualitatively similar
	\item Integrated equilibrium and FPE set are helpful devices here
	\item If $F>G$, then FPE set is ``measure zero'' ($F=2,G=1$ on diagonal of Samuelson's angel diagram)
	\item If $G>F$, then production and trade are indeterminate, but factor content of trade known
\end{itemize}
\begin{center}\includegraphics[width=.6\textwidth]{../images/Krugman1995_fig1_2.png}\end{center}
\end{frame}
% -----------------------------------------
\begin{frame}{High-dimensional predictions}
\begin{itemize}
	\item High-dimensional predictions are not much loved, since they are either weak or unintuitive.
	\item See Ethier (Handbook 1984) for a survey. Lots of comparative statics depend on whether $F$ or $G$ is greater.
\end{itemize}
Stolper-Samuelson in higher dimensions is Jones and Scheinkman (JPE 1977) ``friends and enemies'' result:
\begin{itemize}
	\item SS theorem follows from differentiating zero-profit condition
	\item With arbitrary $F$ and $G$, still true that (no joint production)
	\begin{equation*}
		\hat{p}_g = \sum_f \theta_{fg} \hat{w}_f
	\end{equation*}
	\item Suppose $\hat{p}_1 \leq \dots < \hat{p}_G$. 
	Then there exist $f_1$ and $f_2$ such that 
	\begin{equation*}
		\hat{w}_{f_1} <  \hat{p}_1 \leq \dots < \hat{p}_G < \hat{w}_{f_2}
	\end{equation*}
	\item In uneven cases, cannot always identify ``friends'' and ``enemies''
	\item E.g., in Ricardo-Viner, labor is intermediate, $\hat{p}_1 < \hat{w} < \hat{p}_2$
\end{itemize}
\end{frame}
% -----------------------------------------
\begin{frame}{Heckscher-Ohlin-Vanek Theorem}
\begin{itemize}
	\item Without $G=F$, we have results about factor content of trade rather than goods trade
	\item Define net exports of factor by the vector $T_{F}^{c} = AT^c$, where $A$ is the $F \times G$ matrix of unit factor requirements and $T^c$ is net exports of goods by $c$
	\item Heckscher-Ohlin-Vanek theorem: In any country $c$, net exports of factors satisfy $T_F^c = V^c - s^c V^{\text{world}}$ where $s^c$ is $c$'s share of world income
	\item Countries export factors in which they are relatively abundant: $V^c > s^c V^{\text{world}}$ 
	\item This prediction derives from identical technology, FPE, and homothetic preferences.
	Good luck.
\end{itemize}
\end{frame}
% -----------------------------------------
\begin{frame}{The Ricardo-Roy setup in one slide}
Primitives: 
\begin{itemize}
	\item Technologies $A(\omega,\sigma,\gamma_{A,c})$
	\item Endowments $L(\omega,\gamma_{L,c})$
	\item Demands $D(p,I_c\vert\sigma,\gamma_{D,c})$
\end{itemize}
Equilibrium:
\begin{itemize}
	\item Profit maximization by firms
\begin{align*}
Q(\sigma,c) =&\int_{\Omega} A(\omega,\sigma,\gamma_{A,c})L(\omega,\sigma,c)d\omega \\
\Rightarrow & p(\sigma) \leq \min_{\omega\in\Omega} \{w(\omega,c) / A(\omega,\sigma,\gamma_{A,c}) \} \\
 \Omega(\sigma,c) \equiv & \{\omega\in\Omega:L(\omega,\sigma,c)>0)\}  \subseteq \arg\min_{\omega\in\Omega} \{w(\omega,c) / A(\omega,\sigma,\gamma_{A,c}) \} 
\end{align*}
\item Market clearing
\begin{align*}
\int_{\Sigma} L(\omega,\sigma,c)d\sigma &= L(\omega,\gamma_{L,c}) \quad \forall \omega,c \\
\int_{\mathbb{C}} D(p,I_c\vert\sigma,\gamma_{D,c})dc  &= \int_{\mathbb{C}} Q(\sigma,c) dc \quad \forall \sigma
\end{align*}
\end{itemize}
\end{frame}
% -----------------------------------------
\begin{frame}{Core R-R results}
Cross-sectional predictions
\begin{itemize}
	\item Factor assignments
	\item Factor prices (FPE)
	\item Output quantities (Rybczynski)
\end{itemize}
Comparative statics
\begin{itemize}
	\item Price changes (Stolper-Samuelson)
	\item Endowment and taste changes
\end{itemize}
\end{frame}
% -----------------------------------------
\begin{frame}{Factor assignments}
Taking $p(\sigma)$ as given, who makes what? \\
\textbf{Assumption 1}: 
\textit{$A(\omega,\sigma,\gamma_{A,c})$ is strictly log-supermodular in $(\omega,\sigma)$ and in $(\sigma,\gamma_{A,c})$} \\
Under this assumption:
\begin{itemize}
	\item PAM (I). $\Omega(\sigma,c)$ is increasing in $\sigma$.
	\item[] Recall $\Omega(\sigma,c) \subseteq \arg\min_{\omega\in\Omega} \{w(\omega,c) / A(\omega,\sigma,\gamma_{A,c}) \} $ and property 3 of LSM
	\item PAM(II). $\Sigma(\omega,\gamma_{A,c})$ is increasing in $\gamma_{A,c}$.
	\item[] Similarly, $\Sigma(\omega,\gamma_{A,c}) \subseteq \arg\max_\sigma \{p(\sigma)A(\omega,\sigma,\gamma_{A,c})\}$ and property 3 of LSM
\end{itemize}
Thus, we can say that
\begin{itemize}
	\item High-$\omega$ factors are employed in high-$\sigma$ activities
	\item High-$\gamma_{A,c}$ locations specialize in high-$\sigma$ activities
\end{itemize}
\end{frame}
% -----------------------------------------
\begin{frame}{Factor prices}
\begin{itemize}
	\item Since factors of production are perfect substitutes within each task,
factor price equalization necessarily holds in terms of (Hicks-neutral-adjusted) efficiency units.
\item Just analyze the integrated economy, basically.
\item Factor prices equalize for common technologies:
\begin{align*}
\omega \in \Omega(\sigma,c) \Rightarrow w(\omega,c) = \max_{\sigma} p(\sigma) A(\omega,\sigma,\gamma_{A,c}) \\
\gamma_{A,c} = \gamma_{A,c'} \Rightarrow w(\omega,c)=w(\omega,c')
\end{align*}
\item Assuming a continuum of factors, factor prices are governed by
\begin{align*}
w(\omega,c) = p(M(\omega,c))A(\omega,M(\omega,c)),\gamma_{A,c}) \\
\Rightarrow \frac{d \ln w(\omega,c)}{d\omega} = \frac{\partial \ln A(\omega,M(\omega,c),\gamma_{A,c})}{\partial \omega}
\end{align*}
\end{itemize}
\end{frame}
% -----------------------------------------
\begin{frame}{Output quantities}
We have assignments, but we need a further assumption to get quantities\\
\textbf{Assumption 2}: 
\textit{$L(\omega,\gamma_{L,c})$ is log-supermodular}\\
This means high-$\gamma_{L,c}$ locations are relatively abundant in high-$\omega$ factors
\begin{align*}
Q(\sigma,c) & = \int_{\Omega} A(\omega,\sigma,\gamma_{A,c})L(\omega,\sigma,\gamma_{L,c})d\omega \\
 & = \int_{\Omega(\sigma,c)} A(\omega,\sigma,\gamma_{A,c})L(\omega,\gamma_{L,c})d\omega &\textnormal{by $\Sigma(\omega,c)$ singleton}\\
 & = \int_{\Omega(\sigma)} A(\omega,\sigma,\gamma_{A,c})L(\omega,\gamma_{L,c})d\omega & \textnormal{by common technologies} 
\end{align*}
Rybczynski: Under Assumptions 1 and 2, $Q(\sigma,c)$ is log-supermdular in $(\sigma,\gamma_{L,c})$ \\
Rybczynski + identical preferences $\Rightarrow$ Heckscher-Ohlin theorem
\end{frame}
% -----------------------------------------
\begin{frame}{Price changes}
Consider a small open economy $c$ facing prices $p(\sigma,\phi)$, so that assignments $M(\omega,c,\phi)$ and wages $w(\omega,c,\phi)$ respond to an external change from $\phi$ to $\phi'$.
\\ \textbf{Assumption 3}: \textit{$p(\sigma,\phi)$ is log-supermodular}
\begin{align*}
w(\omega,c,\phi) &= \max_{\sigma} \{p(\sigma,\phi)A(\omega,\sigma,\gamma_{A,c})\}\\
&= p(M(\omega,c,\phi),\phi)A(\omega,M(\omega,c,\phi),\gamma_{A,c}) \\
\Rightarrow \frac{d \ln w(\omega,c,\phi)}{d\phi} &= \frac{\partial \ln p(M(\omega,c,\phi),\phi)}{\partial \phi} \\
\frac{d}{d\omega} \left( \frac{\partial \ln p(M(\omega,c,\phi),\phi)}{\partial \phi} \right) &= \frac{\partial^2 \ln p}{\partial \sigma\partial\phi} \frac{dM}{d\omega} \geq 0
\end{align*}
Assumptions 1 ($A$ LSM) and 3 ($p$ LSM) imply $w(\omega,c,\phi)$ is LSM in $(\omega,\phi)$
\end{frame}
% -----------------------------------------
\begin{frame}[label=comparativestatics]{Taste and endowment changes}
Assignments and wages follow from pair of differential equations
\begin{align*}
\frac{d \ln w(\omega,c)}{d\omega} &= \frac{\partial \ln A(\omega,M(\omega,c),\gamma_{A,c})}{\partial \omega} \\
\frac{d M(\omega,c)}{d\omega} &= \frac{(A(\omega,M(\omega,c),\gamma_{A,c}))^{1-\epsilon}(w(\omega,c))^{\epsilon}L(\omega,\gamma_{L,c})}{B(M(\omega,c),\gamma_{D,c})\int_{\Omega}w(\omega',c)L(\omega',\gamma_{L,c})d\omega'}
\end{align*}
\textbf{Assumption 4}: \textit{$B(\sigma,\gamma_{D,c})$ is log-submodular} \\
Comparative statics under Assumptions 1, 2, 4 for CES demand:
\begin{itemize}
	\item \textbf{Assignments}. $M(\omega,c)$ is decreasing in $\gamma_{D,c}$ and $\gamma_{L,c}$  
	\item\textbf{Prices}. $w(\omega,c)$ is log-submodular in $(\omega,\gamma_{D,c})$ and $(\omega,\gamma_{L,c})$
\end{itemize}
See Costinot \& Vogel (\emph{JPE} 2010) for polarization cases
\end{frame}
% -----------------------------------------
\begin{frame}{}
\begin{center}\LARGE{Empirics}\end{center}
\end{frame}
% -----------------------------------------
\begin{frame}{Trade and regional outcomes (empirics with R-V lens)}
\begin{itemize}
	\item Recent work looking at trade's effects on regional labor markets can be interpreted as using a Ricardo-Viner view
	\item Cross-sectional regressions testing HO model take long-run view, but recent labor literature exploiting panel data lets us take factor specificity more seriously
	\item Suppose a trade-policy change affects $p$ (nationwide goods prices)
	\item What happens to economic outcome in different regions?
	\item \href{https://www.aeaweb.org/articles?id=10.1257/app.2.4.1}{Topalova (2010)} on India,
	\href{https://www.aeaweb.org/articles?id=10.1257/aer.103.5.1960}{Kovak (2013)} on Brazil,
	\href{https://www.aeaweb.org/articles?id=10.1257/aer.103.6.2121}{Autor, Dorn, and Hanson (2013)} on US
\end{itemize}
\end{frame}
% -----------------------------------------
\begin{frame}{Topalova (2010)}
Regression for outcome $y$ in district $d$ in year $t$
	\begin{equation*}
		y_{dt} = \alpha_d^D + \alpha_t^T + \beta \text{tariff}_{dt} + \epsilon_{dt}
	\end{equation*}
\begin{columns}
\begin{column}{.4\textwidth}
\includegraphics[width=\textwidth]{../images/Topalova2010_fig1b.pdf}
\end{column}
\begin{column}{.58\textwidth}
\begin{itemize}
	\item $y$ is poverty rate and tariff is employment-weighted average of national industry import tariffs
	\item India has long-running poverty surveys, many districts, and a large trade liberalization in 1991
	\item IV for tariffs: initial level, because tariff harmonization meant ``the higher the tariff, the bigger the cut''
\end{itemize}
\end{column}
\end{columns}
\end{frame}
% -----------------------------------------
\begin{frame}{}
\begin{center}
\includegraphics[height=\textheight]{../images/Topalova2010_tab3a.pdf}
\end{center}
\end{frame}
% -----------------------------------------
\begin{frame}{Kovak (2013)}
Look at Brazil's import liberalization
\begin{itemize}
	\item Topalova finds little geographical or intersectoral migration
	\item In Brazil, substantial migratory responses
\end{itemize}
Estimating equation explicitly derived from a RV model
\begin{itemize}
	\item Good $i$ with specific factor $K_i$ and labor $L$
	\item Factor market clearing:
	\begin{equation*}
	a_{K_i} Y_i = K_i \qquad \sum_{i} a_{Li} Y_i = L
	\end{equation*}
	\item Differentiating, $\hat{L} = \sum_i \lambda_i (\hat{a}_{Li} - \hat{a}_{K_i})$ where $\lambda_i \equiv L_i / L$
	\item $\hat{p}_i = (1-\theta_i)\hat{w} + \theta_i \hat{r}_i$, where $\theta_i \equiv \frac{r_iK_i}{p_i Y_i}$ is specific factor's cost share
\end{itemize}
\end{frame}
% -----------------------------------------
\begin{frame}{Kovak (2013): Model, continued}
If $\sigma_i$ is elasticity of substitution btw $K_i$ and $L$ then
\begin{equation*}
\hat{a}_{K_i} - \hat{a}_{Li} = \sigma_i \left(\hat{w} - \hat{r}_i\right)
\end{equation*}
Combining with expression for $\hat{L}$, we get
\begin{equation*}
\hat{L} = \sum_i \lambda_i \sigma_i \left( \hat{r}_i - \hat{w}\right)
\end{equation*}
Solve for $\hat{w}$ using some matrix algebra
\begin{equation*}
\hat{w}
=
- \frac{1}{\sum_{i'} \lambda_{i'} \frac{\sigma_{i'}}{\theta_{i'}}} \hat{L}
+ \sum_i \frac{\lambda_{i} \frac{\sigma_{i}}{\theta_{i}}}{\sum_{i'} \lambda_{i'} \frac{\sigma_{i'}}{\theta_{i'}}} \hat{p}_i
\end{equation*}
\vspace{-3mm}
\begin{itemize}
	\item In baseline, no migration, so $\hat{L}=0$
	\item Idiot's law of elasticities says $\sigma_i =1 \ \forall i$
	\item Extend to address non-traded goods
\end{itemize}
Estimate using region's tariff change assuming full passthrough
\begin{equation*}
\Delta \ln w_{r} = \alpha + \beta \cdot \text{RTC}_r + \epsilon_r
\qquad
\text{RTC}_r\equiv \sum_i \frac{\lambda_{i} \frac{1}{\theta_{i}}}{\sum_{i'} \lambda_{i'} \frac{1}{\theta_{i'}}} \Delta \ln \left(1 + \tau_i\right)
\end{equation*}
\end{frame}
% -----------------------------------------
\begin{frame}{Kovak (2013): Identifying variation}
\includegraphics[width=.49\textwidth]{../images/Kovak2013_fig1.pdf}
\includegraphics[width=.49\textwidth]{../images/Kovak2013_fig3.pdf}
\end{frame}
% -----------------------------------------
\begin{frame}{Kovak (2013): Empirical estimates}
\includegraphics[width=\textwidth]{../images/Kovak2013_tab1.pdf}
\href{https://www.aeaweb.org/articles?id=10.1257/aer.20161214}{Dix-Carneiro and Kovak (2017)} estimate dynamic version:
	\begin{equation*}
w_{r,t} - w_{r,\text{1991}} = \alpha_{st} + \beta_t \cdot \text{RTC}_r + \gamma_t (w_{r,\text{1990}} - w_{r,\text{1986}}) +  \epsilon_{rt}
\end{equation*}
\end{frame}
% -----------------------------------------
\begin{frame}{Autor, Dorn, Hanson (2013)}
\begin{itemize}
	\item Use of trade quantities (China shock) rather than prices, so a gravity-based model rather than specific-factor SOE
	\item Exogenous Chinese export supply shock in industry $j$ is $\hat{A}_{Cj}$
	\item Look at region $i$'s outcomes for wages $\hat{w}_i$, employment in traded goods $\hat{L}^{T}_{i}$, and employment in non-traded goods $\hat{L}^{N}_{i}$
	\item Treatment is exposure to import competition (shift-share design):
	\begin{equation*}
		\Delta \text{IPW}_{Uit} = \sum_j \frac{L_{ijt}}{L_{Ujt}} \frac{\Delta M_{UCjt}}{L_{it}}
	\end{equation*}
	\item Instrument using non-US exposure (``other'' $o$):
	\begin{equation*}
		\Delta \text{IPW}_{oit} = \sum_j \frac{L_{ijt-1}}{L_{ujt-1}} \frac{\Delta M_{oCjt}}{L_{it-1}}
	\end{equation*}
\end{itemize}
\end{frame}
% -----------------------------------------
\begin{frame}{ADH (2013): Manufacturing employment falls}
\includegraphics[width=\textwidth]{../images/AutorDornHanson2013_tab2.pdf}
\end{frame}
% -----------------------------------------
\begin{frame}{Autor, Dorn, Hanson (2013): Population response}
\includegraphics[width=\textwidth]{../images/AutorDornHanson2013_tab4.pdf}
\end{frame}
% -----------------------------------------
\begin{frame}{Autor, Dorn, Hanson (2013): Margins of adjustment}
\includegraphics[width=\textwidth]{../images/AutorDornHanson2013_tab5.pdf}
\end{frame}
% -----------------------------------------
\begin{frame}{Empirical work on factor content of trade}
\begin{itemize}
	\item Leontief (1953) and Leamer (1980)
	\item Bowen, Leamer and Sveikauskas (1987)
	\item Trefler (1993)
	\item Trefler (1995)
	\item Davis and Weinstein (2001)
\end{itemize}
\end{frame}
% -----------------------------------------
\begin{frame}{Net factor content of trade}
\begin{itemize}
	\item HOV says net exports of factors are
	$T_{F}^{c} = AT^c  = V^c - s^c V^{\text{world}}$
	\item In reality, production uses intermediate inputs
	\item Let $A(\omega)$ denote input-output matrix for commodity production and $B(\omega)$ denote matrix of direct factor inputs
	\item If only final goods are traded, (Leontief shows that)
	the HOV theorem applies with $\bar{B}(\omega) = B(\omega)\left(I-A(\omega)\right)^{-1}$ in place of $A(\omega)$
	\item Trefler and Zhu (2010) show that the ``only final goods are traded''assumption is not innocuous and propose extensions to address
	\item Also note recent work by Johnson and Noguera on gross vs value-added trade
	\item Prediction of $\bar{B}(\omega) T^c  = V^c - s^c V^{\text{world}}$ is $C \times F$ equalities.
	\item One can imagine many different tests, and there have been.
\end{itemize}
\end{frame}
% -----------------------------------------
\begin{frame}{Leontief (1953)}
\begin{itemize}
	\item Leontief (1953) was the first to empirically examine HOV,
	since Leontief had just computed (for the first time) the input-output table for the 1947 US economy
	\item Leontief's table only had $K$ and $L$ as factor inputs, and he only had $\bar{B}^{\text{US}}(\omega^{\text{US}})$,
	so he assumed $\bar{B}^c(\omega^c) = \bar{B}^{\text{US}}(\omega^{\text{US}}) \ \forall c$
	\item The US had been presumed to be a capital-abundant economy
\end{itemize}
\includegraphics[width=\textwidth]{../images/Feenstra2015_tab2_1.png}
\end{frame}
% -----------------------------------------
\begin{frame}{Leamer (1980): ``The Leontief Paradox, Reconsidered''}
\begin{itemize}
	\item Leontief's intuitively appealing application of HOV to exports and imports was not particularly robust
	\item If trade is unbalanced or $F>2$, then US can be a net exporter of both $K$ and $L$ services
	\item The HOV $T_{F}^{c} = V^c - s^c V^{\text{world}}$ equation implies that
	if $\frac{K^{\text{US}}}{K^{\text{world}}} > \frac{L^{\text{US}}}{L^{\text{world}}}$, then
	\begin{equation*}
		\frac{K^{\text{US}}}{L^{\text{US}}} > \frac{K^{\text{US}} - T_K^{\text{US}}}{L^{\text{US}} - T_L^{\text{US}}} 
	\end{equation*}
	which says capital intensity of production exceeds the capital intensity of consumption
\end{itemize}
\includegraphics[width=\textwidth]{../images/Feenstra2015_tab2_2.png}
\end{frame}
% -----------------------------------------
\begin{frame}{Bowen, Leamer and Sveikauskas (1987)}
\begin{itemize}
	\item Data on 12 factors and 27 countries $\Rightarrow$ many tests of $\bar{B}(\omega) T^c  = V^c - s^c V^{\text{world}}$
	\item Only observe $\bar{B}(\omega)$ in US in 1967, so assume it applies to all countries in 1967
	\item How to test an equality, since it won't hold exactly?
	\begin{itemize}
		\item Sign test: How often does sign of $T_{f}^{c}$ coincide with sign of $V_f^c - s^c V_f^{\text{world}}$? 61\% of the time.
		\item Rank test: If $T_{f}^{c} > T_{f'}^{c}$, is $V_{f}^c - s^c V_{f}^{\text{world}} > V_{f'}^c - s^c V_{f'}^{\text{world}}$? 49\%!
	\end{itemize}
	\item This ``poor performance of the HOV hypothesis'' was disappointing
	\item Maskus (1985) made a similar point in an article titled ``A test of the Heckscher-Ohlin-Vanek theorem: The Leontief commonplace''
	\item BLS (1987) suggest technologies not likely identical
\end{itemize}
\end{frame}
% -----------------------------------------
\begin{frame}{Trefler (1993)}
\begin{itemize}
	\item Trefler (JPE, 1993) and Trefler (AER, 1995) relaxed the strong assumption that technologies were identical
	\item The key was to theoretically incorporate heterogeneous productivities in while keeping attractive properties of HOV
	\item Trefler (1993) introduces country-factor-specific productivity differences $\pi_f^c$ so that effective endowments are $\pi_f^c V_f^c$ and effective factor prices are $w_f^c/\pi_f^c$.
	\item Conditional FPE in terms of efficiency units and HOV similarly
	$$T_{f}^{c} = \bar{B}(\omega) T^c  = \pi^c \cdot V^c - s^c \sum_{c'} \pi^{c'} \cdot V^{c'} $$
	\item There is a (unique) set of $\pi_f^c$ terms that make the HOV equations hold with equality ($\pi_f^{\text{US}}=1$ normalization)
	\item If HOV cannot be wrong, what do we learn from $\pi_f^c$?
\end{itemize}
\end{frame}
% -----------------------------------------
\begin{frame}{Trefler (1993): Sensible $\pi_f^c$?}
\begin{columns}
\begin{column}{.6\textwidth}
\includegraphics[width=\textwidth]{../images/Feenstra2015_fig2_4.png}
\end{column}
\begin{column}{.38\textwidth}
	\begin{itemize}
		\item Trefler shows that estimated $\pi_f^c$ could be negative. Only 10/384 are.
		\item Logic thus far hasn't used factor prices. Conditional FPE looks pretty good.
		\item US is typically more productive ($\pi_f^c<1$)
	\end{itemize}
\end{column}
\end{columns}
\end{frame}
% -----------------------------------------
\begin{frame}{Trefler (1995)}
Trefler (1995) revisits HOV:
\begin{itemize}
	\item Why have factor-content predictions failed? Trefler points to two key sources of failure
	\item What parsimonious extensions of theory (as opposed to the generous $\pi_f^c$) could resurrect HOV with decent fit?
\end{itemize}
First failure:
\begin{itemize}
	\item Define $\epsilon_f^c = T_f^c - (V_f^c - s^c V_f^{\text{world}})$ as HOV deviations
	\item Plot $\epsilon_f^c$ against $V_f^c - s^c V_f^{\text{world}}$: vertical line is perfect fit; diagonal is $T_f^c=0$
	\item 100\% ``sign test'' would put all points in lower left or upper right
	\item Perfect HOV fit is horizontal axis for $\epsilon_f^c=0$
\end{itemize}
\end{frame}
% -----------------------------------------
\begin{frame}{Trefler (1995): ``The Case of the Missing Trade''}
\includegraphics[width=\textwidth]{../images/Trefler1995_fig1.pdf}
\end{frame}
% -----------------------------------------
\begin{frame}{Trefler (1995)}
Second, plot HOV deviations by country
\begin{itemize}
	\item How often is $\epsilon_f^c<0$?
	\item By missing trade, this is mirrored by $V_f^c - s^c V_f^{\text{world}} >0$
\end{itemize}
Visualize a failing of HOV equations:
\begin{itemize}
	\item Poor countries appear to be abundant in all factors.
	\item This cannot be true with balanced trade, and it is not true that poor countries run higher trade imbalances (in Trefler’s sample).
	\item Is there an omitted factor that is scarce in poor countries? Perhaps we need productivity differences again
\end{itemize}
\end{frame}
% -----------------------------------------
\begin{frame}{Trefler (1995): ``Endowments Paradox''}
\includegraphics[width=\textwidth]{../images/Trefler1995_fig2.pdf}
\end{frame}
% -----------------------------------------
\begin{frame}{}
Trefler (1995) considers five parsimonious extensions
\begin{enumerate}
	\item T1: restrict $\pi_f^c$ in Trefler (1993) to $\pi_f^c = \delta^c$ . (`Neutral technology differences').
	\item T2: restrict $\pi_f^c$ in Trefler (1993) to $\pi_f^c = \delta^c \phi_f$ for less developed countries (cutoff estimated) and $\pi_f^c = \delta^c$ for developed countries.
	\item C1: allow the $s^c$ terms to be adjusted to fit the data (variation in countries' demand for investment, services, non-traded goods)
	\item C2: Armington/home bias/trade costs to explain missing trade.
	\item TC2: C2 + $\delta^c=y^c/y^{\text{US}}$
\end{enumerate}
\end{frame}
% -----------------------------------------
\begin{frame}{Trefler (1995): TC2 improves fit considerably}
\includegraphics[width=\textwidth]{../images/Trefler1995_tab1.pdf}
\end{frame}
% -----------------------------------------
\begin{frame}{Gabaix (unpublished, 1997)}
\begin{itemize}
	\item ``Missing trade'' of Trefler (1995) influenced later empirical work on net factor content of trade considerably
	\item Ironically, Gabaix (1997) shows that ``missing trade'' makes the impressive fit of $\pi_f^c$ in Figure 1 of Trefler (1993) less impressive
	\item If trade is entirely missing, $T_f^c=0$, then Trefler (1993) finds $\pi_f^c$ such that $\pi_f^c V_f^c = s^c \sum_{c'} \pi_f^{c'} V_f^{c'}$
	\item If countries are small, this is basically 
	$$ \frac{\pi_f^c}{\pi_f^{c'}} = \frac{Y^c / V_f^c}{Y^{c'} / V_f^{c'}}$$
	relative productivity is relative GDP per factor, so Trefler (1993) fit is unsurprising
\end{itemize}
\end{frame}
% -----------------------------------------
\begin{frame}{Davis and Weinstein (2001)}
\begin{itemize}
	\item Davis, Weinstein, Bradford and Shimpo (1997) found that FCT of trade did reasonably well within Japan, where we may believe FPE is more likely to hold
	\item Davis and Weinstein (2001) seek to understand departures from FPE within the OECD
	\item Prior studies have always applied $\bar{B}(\omega)$ from one country to all others; we don't simply want each country's $\bar{B}^c(\omega)$, since that would make the production side of HOV an identity
	\item Davis and Weinstein (2001) try to parsimoniously parameterize the cross-country differences in $\bar{B}^c(\omega)$ by considering seven nested hypotheses that sequentially drop standard HO assumptions about how endowments affect both \textit{technology} $\bar{B}^c(\cdot)$ and \textit{technique} $\bar{B}^c(\omega)$.
\end{itemize}
\end{frame}
% -----------------------------------------
\begin{frame}{Davis and Weinstein (2001)}
They need to relax a good bit:
\begin{quote}
``a model that allows for technical differences, a breakdown of factor price equalization, the existence of nontraded goods, and costs of trade, is consistent with data for ten OECD countries and a rest-of-world aggregate''
\end{quote}
Hypthoses where ``P'' denotes production specification and ``T'' denotes trade specification
\begin{itemize}
	\item Let $\mathbf{B}^c$ denote the total factor input matrix
	\item $\mathbf{V}^c$ is the endowment vector
	\item $\mathbf{Y}^c$ is the net output vector
\end{itemize}
Assume common technology, free trade, and identical homothetic preferences
$\Rightarrow$ standard HOV theorem
\begin{itemize}
	\item P1: $\mathbf{B}^{\text{US}} \mathbf{Y}^c = \mathbf{V}^c$
	\item T1: 
	$\mathbf{B}^{\text{US}} \mathbf{T}^c = \mathbf{V}^c - s^{c} \mathbf{V}^{\text{world}}$
\end{itemize}
\end{frame}
% -----------------------------------------
\begin{frame}{Davis and Weinstein (2001) hypotheses}
Common technology matrix measured with error: 
$\ln \mathbf{B}^c = \ln \mathbf{B}^{\mu} + \epsilon^c$
\begin{itemize}
	\item P2: $\mathbf{B}^{\mu} \mathbf{Y}^c = \mathbf{V}^c$
	\item T2: 
	$\mathbf{B}^{\mu} \mathbf{T}^c = \mathbf{V}^c - s^{c} \mathbf{V}^{\text{world}}$
\end{itemize}
Hicks-neutral technical differences:
%$\mathbf{B}^c = \lambda^c \mathbf{B}^{\lambda}$ \\ Implement as 
$\mathbf{V}^{cE} = \mathbf{V}^c / \lambda^c$
\begin{itemize}
	\item P3: $\mathbf{B}^{\lambda} \mathbf{Y}^c = \mathbf{V}^{cE}$
	\item T3: 
	$\mathbf{B}^{\lambda} \mathbf{T}^c = \mathbf{V}^{cE} - s^{c} \mathbf{V}^{\text{world}E}$
\end{itemize}
Heterogeneous techniques for traded goods, a la DFS (1980) $\Rightarrow P4, T4$ \\
Violate FPE $\Rightarrow$ heterogeneous techniques for non-traded goods $\Rightarrow P5, T5$ \\
Trade costs $\Rightarrow P6, T6$ \\
\ \\
I'm surely out of time, please look at the paper.
\end{frame}
% -----------------------------------------
\begin{frame}{Next week}
\begin{itemize}
	\item Increasing returns and home-market effects
	\item Print Krugman (1980) and bring to class
\end{itemize}
\end{frame}
% -----------------------------------------
\end{document}
